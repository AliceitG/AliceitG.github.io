\documentclass[conference]{IEEEtran}
\IEEEoverridecommandlockouts
% The preceding line is only needed to identify funding in the first footnote. If that is unneeded, please comment it out.
\usepackage{cite}
\usepackage{amsmath,amssymb,amsfonts}
\usepackage{algorithmic}
\usepackage{graphicx}
\usepackage{textcomp}
\usepackage{xcolor}
\def\BibTeX{{\rm B\kern-.05em{\sc i\kern-.025em b}\kern-.08em
    T\kern-.1667em\lower.7ex\hbox{E}\kern-.125emX}}
\begin{document}

\title{DNA-Computing}

\author{\IEEEauthorblockN{1\textsuperscript{st} Aleyna Acikyol}
\IEEEauthorblockA{\textit{Faculty of Computer Science} \\
\textit{University of Salzburg}\\
Salzburg, Austria \\
s1078913@stud.sbg.ac.at}
\and
\IEEEauthorblockN{2\textsuperscript{nd} Alina Grahic}
\IEEEauthorblockA{\textit{Faculty of Computer Science} \\
\textit{University of Salzburg}\\
Salzburg, Austria \\
-}}

\maketitle

\begin{abstract}
This study serves as an overview of the difficulties of modern data storage and DNA as a possible solution for them as well as computation with DNA. 
\end{abstract}

\begin{IEEEkeywords}
Crash-course DNA, DNA-Computing, Downside, Beginning and Achievement
\end{IEEEkeywords}

\section{Problems of modern computers}
Information has been stored in different ways throughout history. Paintings on cave walls, inscriptions on stone tablets, nowadays books, hard disks, CD’s etc. but also “Cloud” servers.
As a society, we are generating more information each passing day with for example YouTube alone requiring additional 1 million GB in storage per day (source titanpower). Storing all the information with conventional methods will prove difficult in the long run as storage devices usually only last a couple years (source https://graphicardx.com/is-ssd-or-hdd-better-for-long-term-storage/) and space and resources are becoming scarce.
Not only that but transistors on computer chips seem to have reached their limit on how small they can be made (see newatlas). A solution to it are multicore processors yet we will not be able to keep on adding more as for example energy consumption will become a problem. 
So, scientists came up with the idea of using DNA, one of the smallest storages known, to resolve some of the issues. 


\section{Crash-Course DNA}
First of all: What exactly is DNA?
It is short for Desoxyribonucleic Acid, serves as storage for genetic information and is usually found in the nucleus of living cells. 
Structured like a double helix it consists of four organic bases “Adenine”, “Thymine”, “Cytosine” and “Guanine” with two phosphate-sugar molecule backbones also called “Phosphate deoxyribose” Strings. 
The sequence of the bases defines the genetic Information and therefore the code but there is one more thing to note: “Adenine” and “Thymine” always form a pair as well as “Cytosine” and “Guanine” do. Therefore, it is possible to generate binary, ternary, or quaternary code (perhaps even more). 


\section{DNA-Computing}
DNA-Computing is hardware based on DNA, Biochemistry and Molecular biology. But why DNA? DNA is useful for many reasons:

\begin{itemize}
\item Availability: The materials for making DNA can be found almost everywhere.
\item Environmentally friendly: Since only naturally occurring molecules are used it is completely non-toxic and recyclable.
\item Longevity: DNA can last for more than 1000 years under certain conditions. As a computer or data storage.
\item Parallelism: Each DNA String equates to one operation yet many trillion can be operated in a single test tube. 
\item enormous storage capacity: 10 trillion (10\textsuperscript{12}) DNA strings can be contained in 1cm\textsuperscript{3} meaning around 10 terabyte data or for each gram DNA 455 exabyte data can be stored.
\item Security(sourcehttps://resources.infosecinstitute.com/topic/dna-cryptography-and-information-security/) Hiding information in the DNA string in various ways could provide high security.
\item Energy Cost: up to no energy is needed to operate with DNA as enzymes do not require it to work.
\end{itemize}
Yet there are downsides for example the speed. It takes a while for the enzymes to do their job and speeding up the process does not seem to be an option. Therefore, the time to respond, read or write will be significantly longer than conventional computer or storages.


\section{Beginning and Achievemens}

\subsection*{But how was this made possible?}
In 1994 Leonard Adleman presented with the TT-100, a testtube with 100 microliters of DNA filled liquid, a proof-of-concept that DNA could be used for programming by solving the Hamiltonian path problem. His proof was the reason modern achievements like the first fully automated DNA data storage by Microsoft and UW could be made.


\begin{thebibliography}{00}
\bibitem{b1} G. Eason, B. Noble, and I. N. Sneddon, ``On certain integrals of Lipschitz-Hankel type involving products of Bessel functions,'' Phil. Trans. Roy. Soc. London, vol. A247, pp. 529--551, April 1955.
\bibitem{b2} J. Clerk Maxwell, A Treatise on Electricity and Magnetism, 3rd ed., vol. 2. Oxford: Clarendon, 1892, pp.68--73.
\bibitem{b3} I. S. Jacobs and C. P. Bean, ``Fine particles, thin films and exchange anisotropy,'' in Magnetism, vol. III, G. T. Rado and H. Suhl, Eds. New York: Academic, 1963, pp. 271--350.
\bibitem{b4} K. Elissa, ``Title of paper if known,'' unpublished.
\bibitem{b5} R. Nicole, ``Title of paper with only first word capitalized,'' J. Name Stand. Abbrev., in press.
\bibitem{b6} Y. Yorozu, M. Hirano, K. Oka, and Y. Tagawa, ``Electron spectroscopy studies on magneto-optical media and plastic substrate interface,'' IEEE Transl. J. Magn. Japan, vol. 2, pp. 740--741, August 1987 [Digests 9th Annual Conf. Magnetics Japan, p. 301, 1982].
\bibitem{b7} M. Young, The Technical Writer's Handbook. Mill Valley, CA: University Science, 1989.
\end{thebibliography}

\end{document}
